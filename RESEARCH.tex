\documentclass{article}

\usepackage{graphicx}


\begin{document}

\author{Muwonge Julius REG 15/U/8595/PS}
\title{A REPORT ON THE CAUSES SUGAR PRICE INFLATION IN UGANADA TODAY}
\maketitle
\section{Executive summary}{This report is about the causes of sugar price inflation in Uganda today and ideologic resolutions that ordinary people think the government should put in practice to keep the price stable 
}
\section{Introduction}{
It has become a common cry to every Ugandan today whether rich or poor about the daily increasing prices of sugar and unsteady price all over the whole country. In Uganda today a kilogramme of sugar in makerere is at 7000sh while in fort portal is at 6000sh and through my resarch, i hereby explainin to you my findings about the causes according to the ordinally citizens.
}
\section{Body}{
Over the past years it has been observed that sugar prices keep on changing over every financial year. This financial year(2016/17) sugar prices have increased beyond everyone's expectations and here are the findings
}
\section{Findings}{\par
High taxation on sugar by the government.\par
Reduction in sugar plantations.\par
Corrupt government officials that have taken part in purchasing and hiding of sugar.\par
Too much sugar export in outside countries by our sugar factories like META has led to scarcity.\par
Too much consumption of sugar has led to scarcity in surplus.\par
Increased number of immigrants from sudan and other countries.\par
Business games among business men.\par
Inadequate sugar factories.\par
It is museveni(President) who has increased the sugar prices since it is his last reign.\par
The love to make poor people more poor by the government.\par
Planning of the financial upcoming financial year.\par
\includegraphics[width=3cm, height=4cm]{picture}
\includegraphics[width=3cm, height=4cm]{pic}
 }
\section{Conclusion}{
Sugar prices are really high and local people are blaming the government selfishness and its official for the price inflation.
}
\section{References}{
Mr joseph nvunabandi a business man in masanafu.\par
Mrs Nakato regina a shopkeeper in kasubi.

}



\end{document}